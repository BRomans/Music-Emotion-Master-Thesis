\chapter{Preface }
\pagestyle{headings}

In 2019, pushed by the continuous unsatisfactory feelings that haunted my mind, I quit my job as software developer and abandoned a stable situation to pursue a Master's degree in Human-Computer Interaction. I had the strong desire to expand my knowledge and unleash my creativity, but I also wanted to dedicate my efforts to something I truly cared about, that could be meaningful for me and for others. When I stumbled upon Brain-Computer Interfaces, and later Affective Computing, something clicked. I could finally draw a line connecting my technical background in Computer Science with my interests in humanistic subjects like philosophy, psychology and human learning; in other words: cognitive science, the study of the mind and its processes. The passionate people I met in these two years gave an essential contribution in shaping the direction of my studies, and finally a fortuitous encounter on a flight from Paris to Milan in January 2020 set the basis for what later became my graduation project. 
\\
\\
The last decade has seen a wave of renewed attention to the individual needs of people, from the fundamental ones like health and education, up to hobbies, creativity and passions. We want to better understand and take care of ourselves, have control over our mind and body and funnel our energies on what really matters to each of us. Like many others, I have always seen computers as an extension of the human brain, not as a substitute tool for it. Artificial Intelligence is not here to replace human intelligence, but to augment it and express its true potentials. Technology needs to understand us so we can use it to better understand ourselves, and this is the great challenge that Affective Computing and Brain-Computer Interfaces will face in the years to come, possibly disrupting society like many other technological innovations did in the past. Such epochal changes are unpredictable and frightening, but consciously embracing them in advance will reduce the risk of collateral damage caused by misusing these technologies. With this project, I took my first step into these innovative fields and I am determined to responsibly design technologies that can improve each individual's life and, consequently, society itself in the years to come.
\\
\\
I need to acknowledge many people for their direct or indirect contributions that made this project possible. First of all my expanded family, including relatives and close friends, that were always supportive and fueled me with love regardless the physical distance. A special thanks goes to Mannes Poël, that guided my learning process for more than a year and got me passionate about BCI. Another special thanks goes to the crew of the Innovation Lab at myBrain Technologies: Giuseppe Spinelli, whose random encounter on a flight created this beautiful opportunity, Xavier Navarro-Sune, that weekly mentored and reviewed my progresses together with Giuseppe, and Yohan Attal that always found the time to share insightful ideas and comments despite being busy in running a company. I also wanna thank all the other colleagues  at myBrainTechnologies that welcomed me and helped in many organizational steps. Finally, I thank my university colleagues and friends, from Université Paris-Saclay and University of Twente, because in the worst moments we stayed together and cheered each other up, and in the best moments we shared our passions and enjoyed our adventures. 
\\
\\
Enjoy the reading,
\\
\\
Michele






