\chapter{Summary}
\pagestyle{headings}

This research set out to investigate the feasibility of performing Emotion-Recognition using Melomind, a wearable neural interface manufactured by myBrainTechnologies capable of recording \ac{EEG}, in the form of a classification task of the emotional dimensions of valence and arousal.

This study introduces the fields of \ac{BCIs} and Affective Computing, the perception of the market, the leading companies producing wearable \ac{BCIs} for non-clinical applications and the relevance of studying emotions using music, from both the perspectives of market demand and enhancing the user experience.

The goal of this research was to evaluate the feasibility of using Melomind for a future real-time application that can be used to perform Emotion-Recognition. In order to do so, the Valence-Arousal model by James Russel was used as metric for the dimensions of emotions, then several models of emotional correlates in brain activity were evaluated to define what features of the EEG would be more suitable for the task.

The relevant related work was reviewed and studied to provide a methodological framework to the task that could be adapted to the constraints imposed by the limited hardware of the Melomind. An experimental protocol was designed around the inherent advantages of wearable technologies to create a dataset with continuous labelling of emotions on the Valence-Arousal space. Possible biases caused by listening conditions, data labelling tools, emotional affects caused by music, multiple cognitive tasks and external factors were taken into account and tested during a pilot week with employees of myBrainTechnologies.

Data were collected using a robust protocol in two different conditions, eyes-open with a labelling task and eyes-closed. Data were then processed using a lightweight automated preprocessing pipeline and two types of features were extracted from the Power Spectra Density of the EEG signal: neuromarkers and frequency-band specific spectral features calculated in the Theta, Alpha and Beta bands of the EEG signal. Features dimensionality was reduced using Principal Component Analysis and the classification task was performed with subject-dependent strategy. The problem was simplified into two separate binary classifications tasks for valence and arousal, and two supervised learning algorithms were tested: \ac{SVM} and \ac{MLP}. The hyper-parameters were tuned using GridSearch to select the configuration that yielded the highest MCC score, a coefficient that is gaining popularity in machine learning research thanks to its higher reliability. 

All models were then trained and tested using 5-fold LOBO cross-validation that produced two cross-validated scores on the training datasets: CV accuracy and CV MCC. Then, models were further tested on a completely unseen split of data that produced two more scores: test accuracy and MCC. Results were collected and the two classification methods were compared with each other and then with the comparable related work. 

Some models showed promising classification results, reaching ~80\% accuracy in arousal classification and ~75\% accuracy in valence classification with both SVM and MLP, and MCC scores confirmed an average capability to learn. However, the average classification results did not meet the initial expectations and are below  most of the related studies, suggesting that the adopted lightweight pre-processing, the low amount of electrodes of the Melomind or a combination of both hinder the classification task and are not yet suitable for real-time Emotion-Recognition.

The final discussion covers the current challenges of real-time Emotion-Recognition reported by this and related studies and delves into possible improvement of the emotional self-reporting, the features selection, the artifacts cleaning process and the requirements to move from subject-dependent classification to subject independent-classification.

In the conclusion, some considerations are raised from answering the research questions and then an improved artifact cleaning approach is recommended for a follow-up study using the same dataset, that could give further insights on the development of a wearable affective \ac{BCI} using Melomind.
