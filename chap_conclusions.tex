\chapter{Conclusions and recommendations}
\label{chap:conclusions}
The goal of the experiment setup for this research was to answer the main research question: \emph{“What are the performances of classification of music-elicited emotional valence and arousal using a wearable EEG headset?”} and the two sub-research questions that extended it. This research set out to investigate the feasibility of performing \ac{ER} using a wearable interface in the form of a classification task. Data were collected using a robust protocol in two different conditions. They were then processed using a lightweight automated preprocessing pipeline and two types of features were extracted: neuromarkers and frequency-band specific spectral features calculated in the Theta, Alpha and Beta bands of the \ac{EEG} signal. Features dimensionality was reduced using \ac{PCA} and the classification task was performed with subject-dependent strategy. The problem was simplified into two separate binary classifications for valence and arousal, and two algorithms were tested: \ac{SVM} and\ac{MLP}. The hyperparameters were tuned using GridSearch to select the configuration that yielded the highest \ac{MCC} score. All models were then trained and tested using 5-fold \ac{LOBO} cross-validation that produced two cross-validated scores: \ac{CV} accuracy and \ac{CV MCC}. Then, models were further tested on a completely unseen split of data that produced two more scores: test accuracy and \ac{MCC}. Each question is answered separately in the following section and then some recommendations for future work are proposed in the last section.
\section{Conclusions}
\emph{“What are the performances of classification of music-elicited emotional valence and arousal using a wearable EEG headset?”. }

For arousal classification, \ac{SVM} scored higher average test accuracy of 0.63 ± 0.09 and higher average \ac{MCC} score of 0.18 ± 0.2 compared to \ac{MLP} that scored average test accuracy of 0.59 ± 0.09 and average \ac{MCC} score of 0.15 ± 0.17. Cross-validated scores are consistent, with \ac{SVM} scoring higher average CV accuracy of 0.61 ± 0.06 and higher \ac{CV MCC} of 0.24 ± 0. 13, while \ac{MLP} scored average CV accuracy of 0.57 ± 0.07 and average \ac{CV MCC} of 0.15 ± 0.15. The highest consistent \ac{MCC} score for \ac{SVM} was 0.61 with a CV \ac{MCC} of 0.57, the highest consistent \ac{MC}C score for \ac{MLP} was also 0.61 with a \ac{CV} \ac{MCC} of 0.31. For valence classification, \ac{SVM} scored average test accuracy of 0.65 ± 0.12 and similarly \ac{MLP} scored average test accuracy of 0.65 ± 0.11. \ac{MLP} obtained higher average \ac{MCC} of 0.13 ± 0.15 against \ac{SVM} that scored average \ac{MCC} of 0.10 ± 0.19, however \ac{SVM} obtained higher CV accuracy of 0.61 ± 0.07 and higher \ac{CV} \ac{MCC} of 0.26 ± 0.15 against 0.57 ± 0.08 and 0.15 ± 0.16 respectively for \ac{MLP}. Overall, \ac{SVM} yielded more consistency between test and cross-validated scores. In addition, more under-fitted or over-fitted models were generated using \ac{MLP}, with a total of 6 negative \ac{CV MCC} scores in arousal classification against 0 obtained using \ac{SVM}, and a total of 5 negative \ac{CV MCC} scores in valence classification against 1 obtained using \ac{SVM}.
\\
\\
\emph{“What are the most relevant features to perform the Emotion-Recognition task using Power Spectral Density of the EEG signal?”.}

The presented results were obtained after compressing the features using PCA and the contribution of the individual features to the components was not measured. Neuromarkers and frequency band-specific spectral features showed encouraging results, but also great variability among subjects that requires more study on the causes and possible solutions. No neuromarker or subset of features could be proved to be relevant towards subject-independent classification.
\\
\\
\emph{“What is the most suitable classification strategy using a wearable EEG headset?.”}

For this research it was possible to obtain subjective appreciable results using a subject-dependent strategy. Subject-independent classification was discarded during intermediate experiments due to inability of the model to perform better than default guessing the majority class. Thus, subject-dependent classification strategy is for the moment the most suitable using the current wearable technology. Further investigation using more sophisticated signal processing techniques and approaches to deal with unbalanced datasets might lead to more consistent results and enable the development of better strategies that can be applied in an online system with minimal training and calibration time.



\section{Recommendations}
The recommendations basically answer two questions:
\begin{enumerate}
\item How could the presentated research be improved if it were repeated?
\item What would be the appropriate question(s) for future research starting from what is presented here?  
\end{enumerate} 
Dedicate one paragraph to each seperate recommendation, and clearly distinguish between the two types of recommendations.