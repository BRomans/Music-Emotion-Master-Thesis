\chapter{Conclusions and recommendations}
\label{chap:conclusions}
The goal of the experiment setup for this research was to answer the main research question: \emph{“What are the accuracy and MCC scores of subject-dependent classification of music-elicited emotional valence and arousal in the EEG signal using SVM and MLP algorithms with Melomind?”} and the two sub-research questions that extended it. Each question is answered separately in the following section and then some recommendations for future work are proposed in the last section.
\section{Conclusions}
\label{sec:conclusions}
\emph{“What are the accuracy and MCC scores of subject-dependent classification of music-elicited emotional valence and arousal in the EEG signal using SVM and MLP algorithms with Melomind?”. }

For arousal classification, \ac{SVM} scored higher average test accuracy of \(61\pm9\%\) and higher average \ac{MCC} score of \(0.16\pm0.20\) compared to \ac{MLP} that scored average test accuracy of \(58\pm12\% \) and average \ac{MCC} score of \(0.13\pm0.20\). Cross-validated scores are consistent, with \ac{SVM} scoring higher average CV accuracy of \(61\pm6\%\) and higher \ac{CV MCC} of \(0.24\pm0.12\), while \ac{MLP} scored average CV accuracy of \(58\pm8\%\) and average \ac{CV MCC} of \(0.15\pm0.16\). The highest consistent test accuracy was \(84\%\) with \(0.20\) \ac{MCC} and \( 0.08\pm0.32 \) \ac{CV MCC} score for \ac{SVM}; the highest consistent test accuracy was \(88\%\) with \(0.78\) \ac{MCC} and \( 0.28\pm0.24 \) \ac{CV MCC} score for \ac{MLP}. Similarly, for valence classification, \ac{SVM} scored higher average test accuracy of \(67\pm12\%\) and higher average \ac{MCC} score of \(0.13\pm0.18\) compared to \ac{MLP} that scored average test accuracy of \(65\pm12\% \) and average \ac{MCC} score of \(0.11\pm0.18\). Cross-validated scores are again consistent, with \ac{SVM} scoring higher average CV accuracy of \(61\pm6\%\) and higher \ac{CV MCC} of \(0.26\pm0.13\), while \ac{MLP} scored average CV accuracy of \(56\pm9\%\) and average \ac{CV MCC} of \(0.13\pm0.18\). The highest consistent test accuracy was \(89\%\) with \(0.27\) \ac{MCC} and \( 0.02\pm0.20 \) \ac{CV MCC} score for \ac{SVM}; the highest consistent test accuracy was \(77\%\) with \(0.48\) \ac{MCC} and \( 0.35\pm0.14 \) \ac{CV MCC} score for \ac{MLP}. Overall, \ac{SVM} models yielded more consistency between test and cross-validated scores: only 4 and 9 \ac{SVM} models over-fitted for arousal and valence classification respectively, against 6 overfitting and 8 underfitting models in arousal classification and 4 overfitting and 8 underfitting models for valence classification using \ac{MLP}.
\\
\\
\emph{“What are the most relevant selected Power Spectral Density features to perform the Emotion-Recognition using SVM and MLP algorithms with Melomind?”.}

Intermediate experiments selecting features with \ac{SFS} suggested that the neuromarkers were more relevant than other raw features of the EEG signal for a number of participants for subject-dependent classification, but not for the entire population, thus the choice of using \ac{PCA} instead.
The final results were obtained after compressing the features using \ac{PCA} and the contribution of the individual features to the components was not measured. This aggregation of neuromarkers and frequency band-specific spectral features showed encouraging results, but also great variability among subjects that will require more study on the causes and possible solutions to mitigate this effect.  No neuromarker or subset of features could be proved to be relevant towards subject-independent classification.
\\
\\
\emph{“What is the best classification strategy applicable to the current software and hardware capabilities of Melomind using SVM and MLP algorithms?”}

For this research it was possible to obtain subjective appreciable results using a subject-dependent strategy. Subject-independent classification was discarded during intermediate experiments due to inability of the model to perform better than default guessing the majority class. Thus, subject-dependent classification strategy is for the moment the most suitable using Melomind. Further investigation using more sophisticated signal processing techniques and approaches to deal with unbalanced datasets might lead to more consistent results and enable the development of better strategies that can be applied in an online system with minimal training and calibration time.



\section{Recommendations}
\label{sec:recommendations}
Over the course of this research, I faced many design and technical challenges, but the one that took most time to be handled was preparing the data for classification. The preprocessing of the data required continuous reiteration and visualization to understand what was happening when applying certain tools for improving the quality of the \ac{EEG} signal. Ultimately, tools for artifact cleaning like \ac{ASR} were discarded because it was not possible to train them in removing the artifacts without affecting also the "good" segments of signal. The choice of excluding or slicing data was necessary to proceed with the real classification experiment, but took an heavy toll on the dataset and left an unsatisfactory feeling of incompleteness. The \ac{AuPP} is the part of the project that required most time to be developed and tuned and yet it is also the most fragile and the first one that should be rewritten and improved almost entirely. The dataset collected for this research has still a lot of potential information to give and the first step in that direction can be obtained by cleaning the artifacts and including as many datasets as possible. Several types of artifacts are often present in the \ac{EEG} signal \cite{tandle_classification_nodate}:
\begin{itemize}
\item External artifacts: noise caused by interference of other electronic devices, like power-line noise, smartphone frequencies, bad electrodes positioning, electrodes movement.
\item Muscle artifacts: caused by the movement of facial muscles (tongue movement, swallowing) or neck muscles and often appear in frontal and temporal lobes recording.
\item Cardiac artifacts: the heart activity can also be detected by \ac{EEG} electrodes, especially in the left temporal region.
\item Physiological artifacts: these type of artifacts include eyes and eyelids movements and eye blinks and are prominent in the fronto-parietal areas.
\end{itemize}
There are standard signal processing methods to deal with these artifacts, such as filtering (notch and band-pass) to remove interfering electric frequencies, while \ac{PCA} and \ac{ICA} can be used to decompose the signal into components to identify almost any type of artifact, but are only suitable for offline analysis and with a large number of electrodes. Recent researches focusing on automated removal necessarily require to use classification or regression to continuously identify when a segment of signal contains which type of artifact and then apply the right correction. Yeh Sai et al. \cite{sai_automated_2018} performed artifacts identification and removal with wavelet-ICA without visual inspection using a pre-trained \ac{SVM} classifier trained on data contaminated by eye blink artifacts. Their approach allowed the successful removal of target artifacts while retaining most of the \ac{EEG} source signals of interest. A very recent deep learning approach by Rajabioun et al. \cite{rajabioun_deep_2021} was able to successfully classify up to 7 different types of artifacts with \(78.12\%\) accuracy score using \ac{CNN} on a dataset collected to include: blinks, eyes movements, eyebrows movements, head movements, jaw clinches and jaw movements. The effort of cleaning artifacts from \ac{EEG} signal is a common struggle for researchers in academia or neurotech companies and the development of novel methods that can automatically handle artifacts is becoming essential for the design of interactive \ac{BCI} applications that fulfill modern user experience requirements. Considering the aim of this research to evaluate Melomind, a wearable technology with limited hardware, for real-time applications it is clear that this is not only the biggest challenge, but also the one affecting all the other directions for future research and developments. In In conclusion, my primary recommendation for a follow-up study is the collection of contaminated data for the development of cleaning tools for automatic preprocessing of the \ac{EEG} signal that could be then applied to the dataset collected for this research and all future datasets that will be collected using Melomind or similar wearabled devices with dry electrodes. Besides artifacts cleaning, this research on affective \ac{BCI} offers various insights and ideas for further exploring the field of Affective Computing as discussed in Chapter \ref{sec:future_research}. In the time assigned to the project it was only possible to scratch the surface, but hopefully an interesting and actual overview of the state-of-art methods, devices, and strategies was provided to the readers. Researchers and designers interested in building the affective technologies of tomorrow are welcomed to take part in this challenging and exciting world, and share their ideas, insights and solutions with the passionate community that is growing around Brain-Computer Interfaces, that can only exist thanks to the shared efforts, ideas and cooperation that I had the pleasure of discovering during my last two years of experience as scholar of Human-Computer Interaction and Design.
